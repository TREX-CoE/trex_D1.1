\section*{Context}

The context variable is a handle for the state of the library,
and is stored in a data structure which can't be seen outside of
the library.  To simplify compatibility with other languages, the
pointer to the internal data structure is converted into a 64-bit
signed integer, defined in the \texttt{qmckl\_context} type.
A value of \texttt{QMCKL\_NULL\_CONTEXT} for the context is equivalent to a
\texttt{NULL} pointer.

\begin{minted}[frame=lines,fontsize=\scriptsize,linenos]{c}
typedef int64_t qmckl_context ;
#define QMCKL_NULL_CONTEXT (qmckl_context) 0
\end{minted}

\subsection{Context handling}
\label{sec:org502090e}

The context appears as an immutable data structure: modifying a
context returns a new context with the modifications.  Therefore, it
is necessary to store a pointer to the old version of context so
that it can be freed when necessary.
Note that we also provide a possibility to mutate the context, but
this should be done with caution, only when it is justified.

By convention, in this file \texttt{context} is a \texttt{qmckl\_context} variable
and \texttt{ctx} is a \texttt{qmckl\_context\_struct*} pointer.

\subsubsection{Data structure}
\label{sec:orgca6cd07}

The main data structure contains pointers to other data structures,
containing the data specific to each given domain, such that the
modified contexts don't need to duplicate the data but only the
pointers.

\begin{minted}[frame=lines,fontsize=\scriptsize,linenos]{c}
typedef struct qmckl_context_struct {

  /* Pointer to the previous context, before modification */
  struct qmckl_context_struct * prev;

  /* Molecular system */
  qmckl_ao_basis_struct    * ao_basis;

  /* To be implemented:
  qmckl_nucleus_struct     * nucleus;
  qmckl_electron_struct    * electron;
  qmckl_mo_struct          * mo;
  qmckl_determinant_struct * det;
  */

  /* Numerical precision */
  qmckl_precision_struct   * fp;

  /* Error handling */
  qmckl_error_struct       * error;

  /* Memory allocation */
  qmckl_memory_struct      * alloc;

  /* Thread lock */
  int                        lock_count;
  pthread_mutex_t            mutex;

  /* Validity checking */
  uint32_t                   tag;

} qmckl_context_struct;
\end{minted}


A tag is used internally to check if the memory domain pointed
by a pointer is a valid context. This allows to check that even if
the pointer associated with a context is non-null, we can still
verify that it points to the expected data structure.

\begin{minted}[frame=lines,fontsize=\scriptsize,linenos]{c}
#define VALID_TAG   0xBEEFFACE
#define INVALID_TAG 0xDEADBEEF
\end{minted}

The \texttt{qmckl\_context\_check} function checks if the domain pointed by
the pointer is a valid context. It returns the input \texttt{qmckl\_context}
if the context is valid, \texttt{QMCKL\_NULL\_CONTEXT} otherwise.

\begin{minted}[frame=lines,fontsize=\scriptsize,linenos]{c}
qmckl_context qmckl_context_check(const qmckl_context context) ;
\end{minted}

\begin{minted}[frame=lines,fontsize=\scriptsize,linenos]{c}
qmckl_context qmckl_context_check(const qmckl_context context) {

  if (context == QMCKL_NULL_CONTEXT)
    return QMCKL_NULL_CONTEXT;

  const qmckl_context_struct* ctx = (qmckl_context_struct*) context;

  if (ctx->tag != VALID_TAG)
    return QMCKL_NULL_CONTEXT;

  return context;
}
\end{minted}

\subsubsection{Creation}
\label{sec:orgd8b711a}

To create a new context, \texttt{qmckl\_context\_create()} should be used.
\begin{itemize}
\item Upon success, it returns a pointer to a new context with the \texttt{qmckl\_context} type
\item It returns \texttt{QMCKL\_NULL\_CONTEXT} upon failure to allocate the internal data structure
\end{itemize}

\begin{minted}[frame=lines,fontsize=\scriptsize,linenos]{c}
qmckl_context qmckl_context_create() {

  qmckl_context_struct* ctx =
    (qmckl_context_struct*) qmckl_malloc (QMCKL_NULL_CONTEXT, sizeof(qmckl_context_struct));

  if (ctx == NULL) {
    return QMCKL_NULL_CONTEXT;
  }

  /* Set all pointers to NULL */
  memset(ctx, 0, sizeof(qmckl_context_struct));

  /* Initialize lock */
  init_lock(&(ctx->mutex));

  /* Initialize data */
  ctx->tag = VALID_TAG;

  const qmckl_context context = (qmckl_context) ctx;
  assert ( qmckl_context_check(context) != QMCKL_NULL_CONTEXT );

  return context;
}
\end{minted}
\subsection{Access to the previous context}
\label{sec:org80592fe}

\texttt{qmckl\_context\_previous} returns the previous context. It returns
\texttt{QMCKL\_NULL\_CONTEXT} for the initial context and for the \texttt{NULL} context.

\begin{minted}[frame=lines,fontsize=\scriptsize,linenos]{c}
qmckl_context qmckl_context_previous(const qmckl_context context) {

  const qmckl_context checked_context = qmckl_context_check(context);
  if (checked_context == (qmckl_context) 0) {
    return (qmckl_context) 0;
  }

  const qmckl_context_struct* ctx = (qmckl_context_struct*) checked_context;
  return qmckl_context_check((qmckl_context) ctx->prev);
}
\end{minted}
\subsubsection{Locking}
\label{sec:orge3ea316}

For thread safety, the context may be locked/unlocked. The lock is
initialized with the \texttt{PTHREAD\_MUTEX\_RECURSIVE} attribute, and the
number of times the thread has locked it is saved in the
\texttt{lock\_count} attribute.

\begin{minted}[frame=lines,fontsize=\scriptsize,linenos]{c}
void init_lock(pthread_mutex_t* mutex) {
  pthread_mutexattr_t attr;
  int rc;
  
  rc = pthread_mutexattr_init(&attr); 
  assert (rc == 0);
  
  (void) pthread_mutexattr_settype(&attr, PTHREAD_MUTEX_RECURSIVE);
  
  rc = pthread_mutex_init ( mutex, &attr);
  assert (rc == 0);
  
  (void)pthread_mutexattr_destroy(&attr);
}

void qmckl_lock(qmckl_context context) {
  if (context == QMCKL_NULL_CONTEXT)
    return ;
  qmckl_context_struct *ctx = (qmckl_context_struct*) context;
  errno = 0;
  int rc = pthread_mutex_lock( &(ctx->mutex) );
  if (rc != 0) {
    fprintf(stderr, "qmckl_lock:%s\n", strerror(rc) );
    fflush(stderr);
  }
  assert (rc == 0);
  ctx->lock_count++;
/*
  printf("  lock : %d\n", ctx->lock_count);
*/
}

void qmckl_unlock(qmckl_context context) {
  qmckl_context_struct *ctx = (qmckl_context_struct*) context;
  int rc = pthread_mutex_unlock( &(ctx->mutex) );
  if (rc != 0) {
    fprintf(stderr, "qmckl_unlock:%s\n", strerror(rc) );
    fflush(stderr);
  }
  assert (rc == 0);
  ctx->lock_count--;
/*
  printf("unlock : %d\n", ctx->lock_count);
*/
}
\end{minted}

\subsubsection{Copy}
\label{sec:org38bfb02}

\texttt{qmckl\_context\_copy} makes a shallow copy of a context. It returns
\texttt{QMCKL\_NULL\_CONTEXT} upon failure.

\begin{minted}[frame=lines,fontsize=\scriptsize,linenos]{c}
qmckl_context qmckl_context_copy(const qmckl_context context) {

  qmckl_lock(context);

  const qmckl_context checked_context = qmckl_context_check(context);

  if (checked_context == QMCKL_NULL_CONTEXT) {
    qmckl_unlock(context);
    return QMCKL_NULL_CONTEXT;
  }

  
  qmckl_context_struct* old_ctx =
    (qmckl_context_struct*) checked_context;

  qmckl_context_struct* new_ctx =
    (qmckl_context_struct*) qmckl_malloc (context, sizeof(qmckl_context_struct));

  if (new_ctx == NULL) {
    qmckl_unlock(context);
    return QMCKL_NULL_CONTEXT;
  }

  /* Copy the old context on the new one */
  memcpy(new_ctx, old_ctx, sizeof(qmckl_context_struct));

  new_ctx->prev = old_ctx;

  qmckl_unlock( (qmckl_context) new_ctx );
  qmckl_unlock( (qmckl_context) old_ctx );
  
  return (qmckl_context) new_ctx;
}

\end{minted}
\subsubsection{Destroy}
\label{sec:org55cdc30}

The context is destroyed with \texttt{qmckl\_context\_destroy}, leaving the ancestors untouched.
It frees the context, and returns the previous context.

\begin{minted}[frame=lines,fontsize=\scriptsize,linenos]{c}
qmckl_context qmckl_context_destroy(const qmckl_context context) {

  const qmckl_context checked_context = qmckl_context_check(context);
  if (checked_context == QMCKL_NULL_CONTEXT) return QMCKL_NULL_CONTEXT;

  qmckl_lock(context);
  
  qmckl_context_struct* ctx = (qmckl_context_struct*) context;
  assert (ctx != NULL);  /* Shouldn't be true because the context is valid */

  qmckl_unlock(context);

  const qmckl_context prev_context = (qmckl_context) ctx->prev;
  if (prev_context == QMCKL_NULL_CONTEXT) {
    /* This is the first context, free all memory. */
    struct qmckl_memory_struct* new = NULL;
    while (ctx->alloc != NULL) {
       new = ctx->alloc->next;
       free(ctx->alloc->pointer);
       ctx->alloc->pointer = NULL;
       free(ctx->alloc);
       ctx->alloc = new;
    }
  }
  
  qmckl_exit_code rc;
  rc = qmckl_context_remove_memory(context,ctx);

  ctx->tag = INVALID_TAG;

  const int rc_destroy = pthread_mutex_destroy( &(ctx->mutex) );
  if (rc_destroy != 0) {
     fprintf(stderr, "qmckl_context_destroy: %s (count = %d)\n", strerror(rc_destroy), ctx->lock_count);
     abort();
  }

  rc = qmckl_free(context,ctx);
  assert (rc == QMCKL_SUCCESS);

  return prev_context;
}
\end{minted}
\subsection{Memory allocation handling}
\label{sec:orgefbc5fb}

\subsubsection{Data structure}
\label{sec:orgdddff50}

Pointers to all allocated memory domains are stored in the context,
in a linked list. The size is also stored, to enable the
computation of the amount of currently used memory by the library.

\begin{minted}[frame=lines,fontsize=\scriptsize,linenos]{c}
typedef struct qmckl_memory_struct {
  struct qmckl_memory_struct * next    ;
  void                       * pointer ;
  size_t                       size    ;
} qmckl_memory_struct;
\end{minted}

\subsubsection{Append memory}
\label{sec:orgbdc2712}

The following function, called in \href{./qmckl\_memory.html}{\texttt{qmckl\_memory.c}}, appends a new
pair (pointer, size) to the data structure.
It is forbidden to pass the \texttt{NULL} pointer, or a zero size.
If the context is \texttt{QMCKL\_NULL\_CONTEXT}, the function returns
immediately with \texttt{QMCKL\_SUCCESS}.

\begin{minted}[frame=lines,fontsize=\scriptsize,linenos]{c}
qmckl_exit_code qmckl_context_append_memory(qmckl_context context,
                                            void* pointer,
                                            const size_t size) {
  assert (pointer != NULL);
  assert (size > 0L);

  qmckl_lock(context);

  if ( qmckl_context_check(context) == QMCKL_NULL_CONTEXT ) {
    qmckl_unlock(context);
    return QMCKL_SUCCESS;
  }

  qmckl_context_struct* ctx = (qmckl_context_struct*) context; 

  qmckl_memory_struct* new_alloc = (qmckl_memory_struct*)
    malloc(sizeof(qmckl_memory_struct));

  if (new_alloc == NULL) {
    qmckl_unlock(context);
    return QMCKL_ALLOCATION_FAILED;
  }
  
  new_alloc->next    = NULL;
  new_alloc->pointer = pointer;
  new_alloc->size    = size;

  qmckl_memory_struct* alloc = ctx->alloc;
  if (alloc == NULL) {
    ctx->alloc = new_alloc;
  } else {
    while (alloc != NULL) {
      alloc = alloc->next;
    }
    alloc->next = new_alloc;
  }
    
  qmckl_unlock(context);

  return QMCKL_SUCCESS;

}
\end{minted}

\subsubsection{Remove memory}
\label{sec:org3fef27a}

The following function, called in \href{./qmckl\_memory.html}{\texttt{qmckl\_memory.c}}, removes a 
pointer from the data structure.
It is forbidden to pass the \texttt{NULL} pointer.
If the context is \texttt{QMCKL\_NULL\_CONTEXT}, the function returns
immediately with \texttt{QMCKL\_SUCCESS}.

\begin{minted}[frame=lines,fontsize=\scriptsize,linenos]{c}
qmckl_exit_code qmckl_context_remove_memory(qmckl_context context,
                                            const void* pointer) {
  assert (pointer != NULL);

  qmckl_lock(context);

  if ( qmckl_context_check(context) == QMCKL_NULL_CONTEXT ) {
    qmckl_unlock(context);
    return QMCKL_SUCCESS;
  }

  qmckl_context_struct* ctx = (qmckl_context_struct*) context; 

  qmckl_memory_struct* alloc = ctx->alloc;
  qmckl_memory_struct* prev  = ctx->alloc;

  while ( (alloc != NULL) && (alloc->pointer != pointer) ) {
    prev = alloc;
    alloc = alloc->next;
  }
  
  if (alloc != NULL) {
    prev->next = alloc->next;
    free(alloc);
  } 
      
  qmckl_unlock(context);

  if (alloc != NULL) {
    return QMCKL_SUCCESS;
  } else {
    return QMCKL_DEALLOCATION_FAILED;
  }
}
\end{minted}

\subsection{Error handling}
\label{sec:orgd61ab50}

\subsubsection{Data structure}
\label{sec:org31be6ee}

\begin{minted}[frame=lines,fontsize=\scriptsize,linenos]{c}
#define  QMCKL_MAX_FUN_LEN   256
#define  QMCKL_MAX_MSG_LEN  1024

typedef struct qmckl_error_struct {

  qmckl_exit_code exit_code;
  char function[QMCKL_MAX_FUN_LEN];
  char message [QMCKL_MAX_MSG_LEN];

} qmckl_error_struct;
\end{minted}

\subsubsection{Updating errors}
\label{sec:orgb2ce614}

The error is updated in the context using
\texttt{qmckl\_context\_update\_error}, although it is recommended to use
\texttt{qmckl\_context\_set\_error} for the immutable variant.
When the error is set in the context, it is mandatory to specify
from which function the error is triggered, and a message
explaining the error. The exit code can't be \texttt{QMCKL\_SUCCESS}.

\begin{minted}[frame=lines,fontsize=\scriptsize,linenos]{c}
qmckl_exit_code
qmckl_context_update_error(qmckl_context context,
                           const qmckl_exit_code exit_code,
                           const char* function_name,
                           const char* message)
{
  /* Passing a function name and a message is mandatory. */
  assert (function_name != NULL);
  assert (message != NULL);

  /* Exit codes are assumed valid. */
  assert (exit_code >= 0);
  assert (exit_code != QMCKL_SUCCESS);
  assert (exit_code < QMCKL_INVALID_EXIT_CODE);

  qmckl_lock(context);
  
  /* The context is assumed to exist. */
  assert (qmckl_context_check(context) != QMCKL_NULL_CONTEXT);

  qmckl_context_struct* ctx = (qmckl_context_struct*) context;
  assert (ctx != NULL); /* Impossible because the context is valid. */

  if (ctx->error != NULL) {
    free(ctx->error);
    ctx->error = NULL;
  }

  qmckl_error_struct* error =
    (qmckl_error_struct*) qmckl_malloc (context, sizeof(qmckl_error_struct));
  error->exit_code = exit_code;
  strcpy(error->function, function_name);
  strcpy(error->message, message);

  ctx->error = error;

  qmckl_unlock(context);
  
  return QMCKL_SUCCESS;
}
\end{minted}

The \texttt{qmckl\_context\_set\_error} function returns a new context with
the error domain updated.

\begin{minted}[frame=lines,fontsize=\scriptsize,linenos]{c}
qmckl_context
qmckl_context_set_error(qmckl_context context,
                        const qmckl_exit_code exit_code,
                        const char* function_name,
                        const char* message)
{
  /* Passing a function name and a message is mandatory. */
  assert (function_name != NULL);
  assert (message != NULL);

  /* Exit codes are assumed valid. */
  assert (exit_code >= 0);
  assert (exit_code != QMCKL_SUCCESS);
  assert (exit_code < QMCKL_INVALID_EXIT_CODE);

  /* The context is assumed to be valid */
  if (qmckl_context_check(context) == QMCKL_NULL_CONTEXT)
    return QMCKL_NULL_CONTEXT;

  qmckl_context new_context = qmckl_context_copy(context);

  /* Should be impossible because the context is valid */
  assert (new_context != QMCKL_NULL_CONTEXT);

  if (qmckl_context_update_error(new_context,
                                 exit_code,
                                 function_name,
                                 message) != QMCKL_SUCCESS) {
    return context;
  }

  return new_context;
}
\end{minted}


To make a function fail, the \texttt{qmckl\_failwith} function should be
called, such that information about the failure is stored in
the context. The desired exit code is given as an argument, as
well as the name of the function and an error message. The return
code of the function is the desired return code.

\begin{minted}[frame=lines,fontsize=\scriptsize,linenos]{c}
qmckl_exit_code qmckl_failwith(qmckl_context context,
                               const qmckl_exit_code exit_code,
                               const char* function,
                               const char* message) {

  assert (exit_code > 0);
  assert (exit_code < QMCKL_INVALID_EXIT_CODE);
  assert (function != NULL);
  assert (message  != NULL);
  assert (strlen(function) < QMCKL_MAX_FUN_LEN);
  assert (strlen(message)  < QMCKL_MAX_MSG_LEN);

  if (qmckl_context_check(context) == QMCKL_NULL_CONTEXT)
    return QMCKL_NULL_CONTEXT;
  
  const qmckl_exit_code rc = 
    qmckl_context_update_error(context, exit_code, function, message);

  assert (rc == QMCKL_SUCCESS);

  return exit_code;
}

\end{minted}

For example, this function can be used as
\begin{minted}[frame=lines,fontsize=\scriptsize,linenos]{c}
if (x < 0) {
  return qmckl_failwith(context,
                        QMCKL_INVALID_ARG_2,
                        "qmckl_function", 
                        "Expected x >= 0");
 }
\end{minted}

\subsection{Control of the numerical precision}
\label{sec:org7e9d10b}

Controlling numerical precision enables optimizations. Here, the                                                             
default parameters determining the target numerical precision and                                                            
range are defined.                                                                                                           

\begin{table}[htbp]
\label{tab:orga45f0a8}
\centering
\begin{tabular}{lr}
\texttt{QMCKL\_DEFAULT\_PRECISION} & 53\\
\texttt{QMCKL\_DEFAULT\_RANGE} & 11\\
\end{tabular}
\end{table}

\begin{minted}[frame=lines,fontsize=\scriptsize,linenos]{python}
""" This script generates the C and Fortran constants from the org-mode table.
"""

result = [ "#+begin_src c :comments org :tangle (eval h)" ]
for (text, code) in table:
    text=text.replace("~","")
    result += [ f"#define  {text:30s} {code:d}" ]
result += [ "#+end_src" ]

result += [ "" ]

result += [ "#+begin_src f90 :comments org :tangle (eval fh) :exports none" ]
for (text, code) in table:
    text=text.replace("~","")
    result += [ f"   integer, parameter :: {text:30s} = {code:d}" ]
result += [ "#+end_src" ]

return '\n'.join(result)

\end{minted}

\begin{minted}[frame=lines,fontsize=\scriptsize,linenos]{c}
typedef struct qmckl_precision_struct {
  int  precision;
  int  range;
} qmckl_precision_struct;
\end{minted}

The following functions set and get the required precision and
range. \texttt{precision} is an integer between 2 and 53, and \texttt{range} is an
integer between 2 and 11.

The setter functions functions return a new context as a 64-bit
integer. The getter functions return the value, as a 32-bit
integer. The update functions return \texttt{QMCKL\_SUCCESS} or
\texttt{QMCKL\_FAILURE}.

\subsubsection{Precision}
\label{sec:org6cf253c}
\texttt{qmckl\_context\_update\_precision} modifies the parameter for the
numerical precision in a context. If the context doesn't have any
precision set yet, the default values are used.

\begin{minted}[frame=lines,fontsize=\scriptsize,linenos]{c}
qmckl_exit_code qmckl_context_update_precision(const qmckl_context context, const int precision) {

  if (qmckl_context_check(context) == QMCKL_NULL_CONTEXT)
    return QMCKL_INVALID_CONTEXT;

  if (precision <  2) {
    return qmckl_failwith(context,
                          QMCKL_INVALID_ARG_2,
                          "qmckl_context_update_precision",
                          "precision < 2");
  }

  if (precision > 53) {
    return qmckl_failwith(context,
                          QMCKL_INVALID_ARG_2,
                          "qmckl_context_update_precision",
                          "precision > 53");
  }

  qmckl_context_struct* ctx = (qmckl_context_struct*) context;

  /* This should be always true */
  assert (ctx != NULL);

  qmckl_lock(context);

  if (ctx->fp == NULL) {

    ctx->fp = (qmckl_precision_struct*)
      qmckl_malloc(context, sizeof(qmckl_precision_struct));

    if (ctx->fp == NULL) {
      return qmckl_failwith(context,
                            QMCKL_ALLOCATION_FAILED,
                            "qmckl_context_update_precision",
                            "ctx->fp");
    }

    ctx->fp->precision = QMCKL_DEFAULT_PRECISION;
    ctx->fp->range     = QMCKL_DEFAULT_RANGE;
  }

  ctx->fp->precision = precision;

  qmckl_unlock(context);

  return QMCKL_SUCCESS;
}
\end{minted}

\begin{minted}[frame=lines,fontsize=\scriptsize,linenos]{fortran}
  interface
     integer (c_int32_t) function qmckl_context_update_precision(context, precision) bind(C)
       use, intrinsic :: iso_c_binding
       integer (c_int64_t), intent(in), value :: context
       integer (c_int32_t), intent(in), value :: precision
     end function qmckl_context_update_precision
  end interface
\end{minted}

\texttt{qmckl\_context\_set\_precision} returns a copy of the context with a
different precision parameter.

\begin{minted}[frame=lines,fontsize=\scriptsize,linenos]{c}
qmckl_context qmckl_context_set_precision(const qmckl_context context, const int precision) {
  qmckl_context new_context = qmckl_context_copy(context);
  if (new_context == 0) return 0;

  if (qmckl_context_update_precision(new_context, precision) == QMCKL_FAILURE) return 0;

  return new_context;
}
\end{minted}

\texttt{qmckl\_context\_get\_precision} returns the value of the numerical precision in the context.

\begin{minted}[frame=lines,fontsize=\scriptsize,linenos]{c}
int qmckl_context_get_precision(const qmckl_context context) {
  if (qmckl_context_check(context) == QMCKL_NULL_CONTEXT) {
      return qmckl_failwith(context,
                      QMCKL_INVALID_CONTEXT,
                      "qmckl_context_get_precision",
                      "");
  }

  const qmckl_context_struct* ctx = (qmckl_context_struct*) context;
  if (ctx->fp != NULL) 
    return ctx->fp->precision;
  else
    return QMCKL_DEFAULT_PRECISION;
}
\end{minted}

\begin{minted}[frame=lines,fontsize=\scriptsize,linenos]{fortran}
  interface
     integer (c_int32_t) function qmckl_context_get_precision(context) bind(C)
       use, intrinsic :: iso_c_binding
       integer (c_int64_t), intent(in), value :: context
     end function qmckl_context_get_precision
  end interface
\end{minted}

\subsection{Range}
\label{sec:org014eb35}

\texttt{qmckl\_context\_update\_range} modifies the parameter for the numerical range in a given context.

\begin{minted}[frame=lines,fontsize=\scriptsize,linenos]{c}
qmckl_exit_code qmckl_context_update_range(const qmckl_context context, const int range) {

  if (qmckl_context_check(context) == QMCKL_NULL_CONTEXT)
    return QMCKL_INVALID_CONTEXT;

  if (range <  2) {
    return qmckl_failwith(context,
                    QMCKL_INVALID_ARG_2,
                    "qmckl_context_update_range",
                    "range < 2");
  }

  if (range > 11) {
    return qmckl_failwith(context,
                    QMCKL_INVALID_ARG_2,
                    "qmckl_context_update_range",
                    "range > 11");
  }

  qmckl_context_struct* ctx = (qmckl_context_struct*) context;

  /* This should be always true */
  assert (ctx != NULL);

  qmckl_lock(context);

  if (ctx->fp == NULL) {

    ctx->fp = (qmckl_precision_struct*)
      qmckl_malloc(context, sizeof(qmckl_precision_struct));

    if (ctx->fp == NULL) {
      return qmckl_failwith(context,
                      QMCKL_ALLOCATION_FAILED,
                      "qmckl_context_update_range",
                      "ctx->fp");
    }

    ctx->fp->precision = QMCKL_DEFAULT_PRECISION;
    ctx->fp->range     = QMCKL_DEFAULT_RANGE;
  }

  ctx->fp->range = range;

  qmckl_unlock(context);

  return QMCKL_SUCCESS;
}
\end{minted}

\begin{minted}[frame=lines,fontsize=\scriptsize,linenos]{fortran}
  interface
     integer (c_int32_t) function qmckl_context_update_range(context, range) bind(C)
       use, intrinsic :: iso_c_binding
       integer (c_int64_t), intent(in), value :: context
       integer (c_int32_t), intent(in), value :: range
     end function qmckl_context_update_range
  end interface
\end{minted}

\texttt{qmckl\_context\_set\_range} returns a copy of the context with a different precision parameter.

\begin{minted}[frame=lines,fontsize=\scriptsize,linenos]{c}
qmckl_context qmckl_context_set_range(const qmckl_context context, const int range) {
  qmckl_context new_context = qmckl_context_copy(context);
  if (new_context == 0) return 0;

  if (qmckl_context_update_range(new_context, range) == QMCKL_FAILURE) return 0;

  return new_context;
}
\end{minted}

\texttt{qmckl\_context\_get\_range} returns the value of the numerical range in the context.

\begin{minted}[frame=lines,fontsize=\scriptsize,linenos]{c}
int qmckl_context_get_range(const qmckl_context context) {
  if (qmckl_context_check(context) == QMCKL_NULL_CONTEXT) {
      return qmckl_failwith(context,
                      QMCKL_INVALID_CONTEXT,
                      "qmckl_context_get_range",
                      "");
  }

  const qmckl_context_struct* ctx = (qmckl_context_struct*) context;
  if (ctx->fp != NULL) 
    return ctx->fp->range;
  else
    return QMCKL_DEFAULT_RANGE;
}
\end{minted}
\subsection{Helper functions}
\label{sec:org953484b}

\texttt{qmckl\_context\_get\_epsilon} returns \(\epsilon = 2^{1-n}\) where \texttt{n} is the precision.

\begin{minted}[frame=lines,fontsize=\scriptsize,linenos]{c}
double qmckl_context_get_epsilon(const qmckl_context context) {
  const int precision = qmckl_context_get_precision(context);
  return 1. /  (double) (1L << (precision-1));
}
\end{minted}
